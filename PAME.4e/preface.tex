% $Log: preface.tex,v $
% Revision 1.4  2012/01/04 17:20:06  sian
% *** empty log message ***
%
% Revision 1.3  2003-04-23 08:57:32  sian
% Debian release 1.5
%
% Revision 1.2  2002/06/20 11:49:28  sian
% mm removed, padding removed, ca68 added
%
\parindent=5mm
\chapter{Preface}
It is a fallacy to say that progress consists of replacing the workable
by the new.  The brick was invented by the Babylonians and has been
used virtually unchanged for 2500 years.  Even now, despite the advent
of curtain-walling, the brick is still the primary building material.
Likewise, the long-predicted revolution in computer programming to be
produced by the introduction of fourth- and fifth-generation languages
has not come to pass, almost certainly because their purported
advantages are outweighed by their manifest disadvantages. 
Third-generation languages are still used for the bulk of the world's
programming.  Algol~68 has been used as a paradigm of third-generation
languages for 32 years.

Each computer programming language has a primary purpose: C was
developed as a suitable tool in which to write the Unix operating
system, Pascal was designed specifically to teach computer programming
to university students and Fortran was designed to help engineers
perform calculations. Where a programming language is used for its
design purpose, it performs that purpose admirably. Fortran, when it
first appeared, was a massive improvement over assembler languages
which had been used hitherto.  Likewise, C, when restricted to its
original purpose, is an admirable tool, but it is a menace in the hands
of a novice.  However, novices do not write operating systems.

According to the ``Revised Report on the Algorithmic Language Algol
68'' (see the Bibliography), Algol 68 was ``designed to communicate
algorithms, to execute them efficiently on a variety of different
computers, and to aid in teaching them to students''.  Although this
book has not been geared to any specific university syllabus, the
logical development of the exposition should permit its use in such an
environment.  However, since no programming expertise is assumed, the
book is also suitable for home-study.

It is time to take a fresh approach to the teaching of computer
programming.  This book breaks new ground in that direction.  The
concept of a variable (a term borrowed from mathematics, applied to
analogue computers and then, inappropriately, to digital computers) has
been replaced by the principle of value integrity: in Algol~68, every
value is a constant.  All the usual paraphernalia of pointers,
statements and expressions is dispensed with. Instead, a whole new
sublanguage is provided for understanding the nature of programming.

This book covers the language as implemented by the DRA a68toc
translator.  Since the last edition, a new chapter on the Standard
Prelude has been added, thereby bringing together all the references
to that Prelude in the rest of the book. This edition is an interim
edition describing the QAD transput provided with the Algol68toC package.

It has been a conscious aim of the author to reduce the amount of
description to a minimum.  It is advisable, therefore, that the text be
read slowly, re-reading a point if it is not clear.  This is
particularly true for chapter~5 where the concept of the name has been
introduced rather carefully. The exercises are intended to be worked. 
Answers to all the exercises have been given except for those which are
self-marking.

A program written for use with the book can be found in the same
directory as this book.

I should like to thank Wilhelm Kl\"oke for bringing the Algol 68RS
compiler to my attention and James Jones and Greg Nunan for their
active help in the preparation of the QAD transput.

In 40 years of programming, I have had many teachers and mentors, and I
have no doubt that I have benefited from what they have told me,
although now it is difficult to pinpoint precisely which part of my
understanding is due to which individual. Any errors in the book are my
own.  If any reader should feel that the book could be improved, I
should be grateful if she would communicate her suggestions to the
publisher, so that in the event of another edition, I can incorporate
those I feel are appropriate (she includes he).
\bigskip\bigskip
\begin{flushright}
Sian Mountbatten\\
Inbhir Nis\\
Am Mart 2008
\end{flushright}
